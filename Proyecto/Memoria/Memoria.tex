%Encabezado estándar
\documentclass[10pt,a4paper]{article}
\usepackage[utf8]{inputenc}

\usepackage{graphicx}
\graphicspath{ {/home/luissuall/Documentos/Vision_por_Computador/Informe3} }

\usepackage{amsmath}
\usepackage{amssymb}
\usepackage{amsthm}
\usepackage{amsfonts}
\usepackage{enumerate}
\usepackage{listings}
\usepackage{hyperref}

%Configuración para cuando mostremos trozos de codigo:
\lstset{ %
escapeinside=||,
language=C++,
basicstyle=\footnotesize}

\author{Gustavo Rivas Gervilla y Luis Suárez Lloréns}
\title{Memoria}
\date{}

\theoremstyle{definition}
\newtheorem{cuestion}{Cuestión}
\newtheorem*{respuesta}{Respuesta}

\begin{document}
\maketitle

%\includegraphics[scale=0.5]{parte1.jpg}
\section{Nombre del proyecto}

Implementar la creación de panoramas lineales con proyección en superficies cilíndricas o esféricas y mezcla de imágenes usando el algoritmo de Burt-Adelson.\\

\section{Descripción del problema}

Nuestro objetivo es crear panoramas. Esta aplicación se ha visto con anterioridad y se encuentra implementada en diversos dispositivos móviles --- teléfonos inteligentes, cámaras digitales,... --- lo cual hace parecer este proceso algo muy sencillo. Sin embargo, esta aplicación tiene varias dificultades que afrontar para que los resultados sean tan naturales como parecen en las soluciones que podemos encontrar en el mercado actualmente.\\

Por un lado, al trabajar con varias imágenes, necesitamos conocer sus posiciones relativas de una manera completamente automática. Con esto sabremos como colocar las imágenes para poder generar el panorama.\\

Por otro lado, hay que unir las imágenes. Debido a pequeñas variaciones, debidas a factores que varían desde la iluminación hasta posibles errores cometidos en la fase anterior, la superposición de ambas imágenes nos ofrece una imagen poco satisfactoria. La mezcla se debe hacer de una manera más suave, y para ello utilizaremos el algoritmo propuesto por Burt y Adelson.\\

Añadir a este proceso que la proyección que buscamos es sobre superficies cilíndricas o esféricas, lo que nos obliga a realizar alteraciones en las imágenes para conseguir este efecto.\\

\section{Descripción de la solución}

Vamos a separar este apartado en los principales problemas resueltos durante el desarrollo de este proyecto. \\

\subsection{Proyección cilíndrica o esférica}

La gran mayoría de imágenes, son representadas en un plano. Para conseguir obtener la proyección cilíndrica o esférica de dichas imágenes, tendríamos que ver en que punto del cilíndro o esfera corta la línea que hay entre el punto donde se toma la imagen, y el  punto de la proyección original ---imagen original---.\\

Siendo $ (x_{c}, y_{c})$ el centro de la imagen, f la distancia focal y s el llamado radio del cilindro o esfera, las fórmulas para la obtención de las proyecciones son:

\subsubsection{Proyección cilíndrica}
 
 La fórmula para la obtención de los valores es:
 
 \[
 \ x^{'} = s\, tan^{-1} \frac{x - x_{c}}{f} +  x_{c}
 \]
 \[
 \ y^{'} = s\frac{y -y_{c}}{\sqrt{(x +x_{c})^{2}+f^{2}}} + y_{c}
 \]

\subsubsection{Proyección esférica}

La fórmula para la obtención de los valores es:

 \[
 \ x^{'} = s\, tan^{-1} \frac{x - x_{c}}{f} +  x_{c}
 \]
 \[
 \ y^{'} = s\, \tan^{-1} \frac{y -y_{c}}{\sqrt{(x +x_{c})^{2}+f^{2}}} + y_{c}
 \]
\subsection{Construcción del panorama}

Una vez tenemos las imágenes proyectadas, tenemos que reconocer donde estarán colocadas en nuestro panorama.\\

Para ello, hay que tener en cuenta que, al realizar una de estas dos proyecciones, un giro en la cámara se traduce en una traslación de las mismas. Por tanto, dentro de todo el espacio de posibles movimientos, vamos a considerar únicamente las traslaciones horizontales.\\

Con este espacio de búsqueda tan reducido, el cálculo del mejor desplazamiento se simplifica mucho. Optaremos por buscar que traslación nos ofrece un menor error medio. Usamos esta medida del error, pues el número de píxeles que se solapan de las dos imágenes es variable, haciendo que otras medidas ---como la suma de los errores--- sean de poca utilidad. Vamos a buscar pixel a pixel, podríamos considerar otras soluciones que nos podrían dar soluciones con precisión subpixel, pero no requerimos de tanta precisión.\\

Una vez tenemos esto, podríamos simplemente superponer las imágenes. Esto da resultados poco deseables, pues en la imagen obtenida se podían ver claramente una imagen encima de la otra. Para realizar la mezcla de una manera más correcta, utilizamos el algoritmo propuesto por Burt y Adelson.\\

\subsection{Algoritmo Burt-Adelson}

Esta técnica pretende obtener la mezcla suave de dos imágenes. Para explicar esta técnica, vamos a suponer que tenemos dos imágenes que se superponen y queremos mezclar la parte izquierda de una con la derecha de la otra. Más tarde explicaremos como generalizar estas condiciones.\\

El problema de la mezcla de imágenes, como ya hemos comentado anteriormente, es que si simplemente tomamos los píxeles de cada imagen, sin ningún paso más, se generan bordes y otras aberraciones visuales que no queremos obtener. Una primera aproximación al problema sería promediar de alguna manera los píxeles centrales de ambas imágenes. Pese a ser una mejora del algoritmo anterior, también produce efectos no deseados en las imágenes.\\

Estos efectos depende del tamaño que hayamos seleccionado para la franja central que promediamos. Si la franja es pequeña, se sigue apreciando el borde, aunque suavizado. Si la franja es muy grande, se mezclan los detalles de las dos imágenes, deformando el resultado.\\

La solución propuesta por Burt-Adelson propone realizar una mezcla similar a la comentada anteriormente, pero separando primero las imágenes en diferentes bandas de frecuencias, mezclando las bandas y reconstruyendo la imagen. Esto nos ofrece unas imágenes mucho más convincentes.\\

Además, vamos a realizar una versión del algoritmo que permite realizar la mezcla utilizando una máscara para seleccionar que píxeles queremos de cada imagen.\\

\subsubsection{Algoritmo}

El algoritmo, a grandes rasgos, se compone de los siguientes pasos:\\

1) Construcción de las pirámides Laplacianas de las imágenes.\\

Como ya hemos comentado anteriormente, queremos realizar la mezcla de las imágenes para diferentes anchos de banda. Las pirámides Laplacianas son perfectas para esta tarea por dos motivos. Por un lado, divide la imagen en diferentes anchos de banda, paso fundamental para este algoritmo. Y por otro lado, nos facilita a realizar la mezcla, pues genera automáticamente la franja que vamos a mezclar para cada ancho de banda.\\

Veamos esta última propiedad. Los detalles más finos se almacenan en la capa más baja y extensa de la pirámide y la mezcla se realiza en  una franja muy estrecha --- en el algoritmo más básico propuesto por los autores, solo se promedia la columna central, dejando los valores de la imagen izquierda a la izquierda y los de la derecha a la derecha--- . En capas superiores, se realiza de la misma manera, pero la columna central ya no representa una sola columna de píxeles, si no que resume información de varias columnas de la imagen. Entonces, a mayor sea la capa que vayamos a mezclar, tenemos cada vez menos detalles --- amplitud de onda mayor--- y franjas donde mezclar también mayores en términos de la imagen final, justo lo que se quería.\\

2) Construcción de la pirámide Gaussiana de la máscara.\\

Algo similar ocurre con la pirámide Gaussiana de la máscara. La usaremos en el siguiente paso para generar los pesos de cada imagen para cada pixel de una nueva pirámide Laplaciana que será nuestra imagen resultado. Destacar que la estructura de la pirámide Gaussiana --- número de capas y tamaño de las mismas --- encaja perfectamente con las pirámides Laplacianas del paso anterior.

3) Mezcla de las pirámides Laplacianas.\\

Mezclaremos los valores de las pirámides Laplacianas de las imágenes, formando una nueva pirámide Laplaciana. Básicamente, realizaremos una media ponderada de ambas pirámides, usando la pirámide Gaussiana de la máscara como pesos para dicha media ponderada. \\

Notamos $LA$ y $LB$ a las pirámides Laplacianas de las imágenes originales, $LS$ a la pirámide Laplaciana solución y $GM$ a la pirámide Gaussiana de la máscara. Notamos como $L_{k}(i,j)$ al valor de la posición (i,j) en el nivel k. Entonces, la fórmula usada es la siguiente:\\

\[
\ LS_{k}(i,j) = GR_{k}(i,j)LA_{k}(i,j) + (1-GR_{k}(i,j))LB_{k}(i,j)
\]\\

4) Reconstrucción de la imagen.\\

A partir de la pirámide Laplaciana generada en el paso anterior, podemos recuperar una imagen. Dicha imagen es el resultado del algoritmo.\\

\section{Resultados}



\end{document}
